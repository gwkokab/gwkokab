\documentclass{article}
\usepackage[utf8]{inputenc}
\usepackage[T1]{fontenc}
\usepackage{natbib} % For bibliography style
\usepackage{graphicx} % For including figures
\usepackage{amsmath} % For mathematical symbols and equations
\usepackage{hyperref} % For hyperlinks

% Document
\title{Technical Details of Population Inference}
\author{Muhammad Zeeshan}
\author{Meesum}
\date{\today}

\begin{document}

\maketitle
\section{Abstract}
In these notes, we wil discuss the technical details of population inference. 
Specificaly, we will focus on theoretical details and their analytical and numerical 
calculations and then their implementaion in python. 

\section{Introduction}
We will discuss the following topics:
\begin{itemize}
    \item Data Collection or Generation.
    \item Population Models for Mass, Spin, Eccentricity and Redshift.
    \item Volume Time Calculation.
    \item Individual Event and Population Likelihood.
\end{itemize}

    
\section{Data Collection or Generation}

To make a population inference of compact binary coalescence (CBC) events, we need to have a dataset of events.
This dataset can be collected from the publically available data from the LIGO-Virgo Collaboration (LVC) or can 
be generated using a population model.



\section{Conclusion}

In the conclusion section, you can summarize your notes and provide any final remarks.

% Uncomment the following line if you want to include a bibliography
% \bibliographystyle{plain}
% \bibliography{references}

\end{document}
